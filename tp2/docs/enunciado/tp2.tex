\documentclass[11pt, a4paper]{article}
\usepackage{a4wide}
\usepackage{amsmath}
\usepackage{amsfonts}
\usepackage{graphicx}
%\usepackage[ruled,vlined]{algorithm2e}

\parskip = 11pt

\newcommand{\real}{\mathbb{R}}

\begin{document}


\begin{centering}
\large\bf Laboratorio de M\'etodos Num\'ericos - Primer Cuatrimestre 2014 \\
\large\bf Trabajo Pr\'actico N\'umero 2: Tu cara me suena\\
\end{centering}


\vskip 0.5 cm
\hrule
\vskip 0.1 cm

{\bf\noindent Introducci\'on}

A menos de cincuenta d\'ias del inicio de la gran fiesta del f\'utbol mundial, el Equipo de Desarrollos de M\'etodos
Num\'ericos ha recibido un pedido reciente por parte del Comite de Seguridad de la
Organizaci\'on del Mundial Brasil 2014. Informaciones de \'ultimo momento aseguran que una gran cantidad de
\emph{barras} se est\'an organizando para \emph{copar} las tribunas de los (a\'un no terminados) estadios del pa\'is
organizador. Debido al retraso en la implementaci\'on del sistema FIFA Plus por cuestiones t\'ecnicas, el evento ser\'a
cancelado a menos que nuestro equipo logre desarrollar un software de reconocimiento facial efectivo en los pr\'oximos
20 d\'ias.

El Comit\'e Organizador ha puesto a nuestra disposici\'on todos sus recursos a fin de evitar lo que ser\'ia la mayor verg\"uenza
en la historia de la entidad. Entre otras cosas, se dispone de una base de datos de fotograf\'ias digitalizada de todas
aquellas personas que alguna vez hayan ingresado a alg\'un estadio de f\'utbol en alg\'un pa\'is del planeta. El
objetivo del trabajo es desarrollar un software que, tomando esta informaci\'on como entrenamiento, pueda detectar en
tiempo real si los asistentes a alguno de los partidos corresponde a un \emph{barra}.


Como instancias de entrenamiento, se tiene un conjunto de $N$ personas, cada una de ellas con $M$ im\'agenes
(eventualmente) distintas sus rostros en escala de grises del mismo tama\~no y resoluci\'on. Cada una de estas
im\'agenes sabemos a qu\'e persona corresponde.

Para $i = 1,\ldots, n$, sea $x_i \in \real^{m}$ la $i$-\'esima imagen de nuestra base de datos almacenada por filas en
un vector, y sea $\mu = (x_1 + \ldots + x_n)/n$ el promedio de las im\'agenes. Definimos $X\in\real^{n\times m}$ como la
matriz que contiene en la $i$-\'esima fila al vector $(x_i - \mu)^{t}/\sqrt{n-1}$, y $$A=U \Sigma V^t$$ a su descomposici\'on en valores singulares, con $U\in\real^{n\times n}$ y $V\in\real^{m\times m}$ matrices ortogonales, y $\Sigma\in\real^{n\times m}$ la matriz diagonal conteniendo en la posici\'on $(i,i)$ al $i$-\'esimo valor singular $\sigma_i$.
Siendo $v_i$ la columna $i$ de $V$, definimos para $i = 1,\ldots,n$ la \textsl{transformaci\'on caracter\'istica} del d\'igito $x_{i}$ como el vector $\mathbf{tc}(x_i) = (v_1^t\, x_i, v_2^t\, x_i,\ldots,v_k^t\, x_i) \in\real^k$, donde $k \in\{1,\ldots,m\}$ es un par\'ametro de la implementaci\'on. Este proceso corresponde a extraer las $k$ primeras \textit{componentes principales} de cada imagen. La intenci\'on es que $\mathbf{tc}(x_i)$ resuma la informaci\'on m\'as relevante de la imagen, descartando los detalles o las zonas que no aportan rasgos distintivos.

Dada una nueva imagen $x$ de una cara, que no se encuentra en el conjunto inicial de im\'agenes de entrenamiento, el
problema de reconocimiento consiste en determinar a qu\'e persona de la base de datos corresponde. Para esto, se calcula
$\mathbf{tc}(x)$ y se compara con $\mathbf{tc}(x_i)$, para $i = 1,\ldots, n$, utilizando un criterio de selecci\'on
adecuado.


{\bf\noindent Enunciado}

Se pide implementar un programa en \verb+C+ o \verb-C++- que lea desde archivos las im\'agenes de entrenamiento correspondientes a distintas
personas y que, utilizando la descomposici\'on en valores singulares y el n\'umero de componentes principales $k$
mencionado anteriormente, calcule la transformaci\'on caracter\'istica de acuerdo con la descripci\'on anterior. Se debe
proponer e implementar al menos un m\'etodo que, dada una nueva imagen de una cara, determine a que persona de la base
de datos corresponde utilizando la transformaci\'on caracter\'istica.

Con el objetivo de obtener la descomposici\'on en valores singulares, se deber\'a implementar el m\'etodo de la potencia
con deflaci\'on para la estimaci\'on de autovalores/autovectores. En este contexto, la factibilidad de aplicar este
m\'etodo es particularmente sensible al tama\~no de las im\'agenes de la base de datos. Por ejemplo, considerar im\'agenes
en escala de grises de $100 \times 100$ p\'ixeles implicar\'ia trabajar con matrices de tama\~no $10000 \times
10000$. Una alternativa es reducir el tama\~no de las im\'agenes, por ejemplo, mediante un submuestreo. 
Sin embargo, es posible superar esta dificultad en los casos donde el n\'umero de muestras es menor que el
n\'umero de variables. Se pide desarrollar las siguientes sugerencias y fundamentar como utilizarlas en el contexto del
trabajo.

\begin{itemize}
\item Dada una matriz y su descomposici\'on en valores singulares $A = U \Sigma V^t$, encontrar la descomposici\'on en
valores singulares de $A^t$. C\'omo se relacionan los valores singulares de $A$ y $A^t$?
\item Dada la descomposici\'on en valores singulares de $A$, expresar en funci\'on de $U$, $\Sigma$ y $V$ las matrices
$A^t$, $A^tA$ y $AA^t$. Analizar el tama\~no de cada una de ellas y deducir como relacionar las respectivas componentes
principales.
Combinar con el item anterior para el c\'omputo de los componentes principales.
\end{itemize}

En base a este an\'alisis, se pide desarrollar una herramienta alternativa que permita trabajar bajo ciertas condiciones
con im\'agenes de tama\~no mediano/grande.

Junto con este enunciado se provee una base de datos de im\'agenes correspondiente a 41 personas, con 10
im\'agenes por cada una de ellas. Esta base de datos se encuentra disponible en dos resoluciones distintas: $92 \times
112$ y $23 \times 28$ p\'ixeles por cada imagen. La segunda corresponde a un submuestreo de la base original.
En relaci\'on a la experimentaci\'on, se pide como m\'inimo realizar los siguientes experimentos:
\begin{itemize}
\item Analizar para cada una de las variantes qu\'e versi\'on de la base de datos es posible utilizar, en base a
requerimientos de memoria y tiempo de c\'omputo.

\item Para cada una de las variantes propuestas, analizar el impacto en la tasa de efectividad del algoritmo de reconocimiento al
variar la cantidad de componentes principales considerados. Estudiar tambi\'en como impacta la cantidad de im\'agenes
consideradas para cada persona en la etapa de entrenamiento.

\item En caso de considerar m\'as de una posibilidad para determinar a que persona corresponde una nueva cara,
considerar para cada una la mejor configuraci\'on de par\'ametros y compararlas entre ellas.
\end{itemize}

El objetivo final de la experimentaci\'on es proponer una configuraci\'on de par\'armetros/m\'etodos que obtenga
resultados un buen balance entre la tasa de efectividad de reconocimiento de caras, la factibilidad de la propuesta y el
tiempo de c\'omputo requerido.

{\bf Programa y formato de archivos}

Se deber\'an entregar los archivos fuentes que contengan la resoluci\'on del trabajo pr\'actico. El ejecutable tomar\'a
tres par\'ametros por l\'inea de comando, que ser\'an el archivo de entrada, el archivo de salida, y el m\'etodo a
ejectutar (0 m\'etodo est\'andar, 1 m\'etodo alternativo).


Asumimos que la base de datos de im\'agenes se encuentra organizada de la siguiente forma: partiendo de un directorio
ra\'iz, contiene un subdirectorio por cada una de las personas donde se encuentran las im\'agenes de entrenamiento. El
nombre de las im\'agenes ser\'a un n\'umero y su extensi\'on una archivo \verb+.pgm+ (ejemplo: \verb+1.pgm+,
\verb+2.pgm+, etc.). De esta forma se puede saber a que persona corresponde cada imagen simulando un etiquetado de las
mismas. 

Para facilitar la experimentaci\'on, el archivo de entrada con la descripci\'on del experimento sigue la siguiente
estructura:
\begin{itemize}
\item La primera l\'inea contendr\'a el \emph{path} al directorio que contiene la base de datos, seguido de 5 n\'umeros
enteros que representan la cantidad de filas y columnas de las im\'agenes de la base, cuantas personas ($p$) y cuantas
im\'agenes por cada una de ellas ($nimgp$), y cu\'antas componentes principales se utilizar\'an en el experimento ($k$). 
A continuaci\'on de muestra un ejemplo de una base de datos de im\'agenes de $112 \times 92$, con 41 sujetos, 5 
im\'agenes por sujeto y tomando 15 componentes principales.

\begin{verbatim}
../data/ImagenesCaras/ 112 92 41 5 15
\end{verbatim}

\item A continuaci\'on, el archivo contendr\'a $p$ l\'ineas donde en cada una de ellas se especificar\'a la carpeta
correspondiente a la p-\'esima persona, seguido de $nimgp$ numeros enteros indicando las im\'agenes a considerar para el
entrenamiento. La siguiente l\'inea muestra como ejemplo 5 im\'agenes (2, 4, 7, 8 y 10) a considerar para la persona
\verb+s10+.

\begin{verbatim}
s10/ 2 4 7 8 10
\end{verbatim}

\item Finalmente, se especificar\'a un n\'umero $ntest$ de im\'agenes (preferentemente no contenidas en la base de
im\'agenes) para las cuales se desea identificar a quien pertenecen. Cada una de ellas se especificar\'a en una nueva
l\'inea, indicando el \emph{path} al archivo seguido del n\'umero de individuo al que pertenece (relativo a la
numeraci\'on establecida en el punto anterior). La siguiente l\'inea muestra un ejempo de una imagen a testear para el
sujeto n\'umero 1:

\begin{verbatim}
../data/ImagenesCaras/s1/9.pgm 1
\end{verbatim}
\end{itemize}

El archivo de salida obligatorio tendr\'a el vector soluci\'on con los $k$ valores singulares de mayor magnitud, con una
componente del mismo por l\'inea. Adicionalmente, en caso de considerarlo conveniente, se podr\'an agregar archivos
adicionales opcionales con m\'as informaci\'on que la descripta en la presente secci\'on.

Junto con el presente enunciado, se adjunta una serie de scripts hechos en \verb+python+ y un conjunto instancias de
test que deber\'an ser utilizados para la compilaci\'on y un testeo b\'asico de la implementaci\'on. Se recomienda leer
el archivo \verb+README.txt+ con el detalle sobre su utilizaci\'on.


\vskip 0.5 cm
\hrule
\vskip 0.1 cm

{\bf Fecha de entrega} 
\begin{itemize}
\item \textsl{Formato electr\'onico:} Jueves 15 de Mayo de 2014, \underline{hasta las 23:59 hs.}, enviando el trabajo
(informe+c\'odigo) a \texttt{metnum.lab@gmail.com}. El subject del email debe comenzar con el texto \verb|[TP2]| seguido
de la lista de apellidos de los integrantes del grupo. 
\item \textsl{Formato f\'isico:} Viernes 16 de Mayo de 2014, de 17:30 a 18:00 hs.
\end{itemize}


\end{document}
