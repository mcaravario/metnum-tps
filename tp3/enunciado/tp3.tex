\documentclass[11pt, a4paper]{article}
\usepackage{a4wide}
\usepackage{amsmath}
\usepackage{amsfonts}
\usepackage{graphicx}
%\usepackage[ruled,vlined]{algorithm2e}

\parskip = 11pt

\newcommand{\real}{\mathbb{R}}

\begin{document}


\begin{centering}
\large\bf Laboratorio de M\'etodos Num\'ericos - Primer Cuatrimestre 2014 \\
\large\bf Trabajo Pr\'actico N\'umero 3: Hay que poner un poquito mas de \emph esfuerzo...\\
\end{centering}


\vskip 0.5 cm
\hrule
\vskip 0.1 cm

{\bf\noindent Introducci\'on}

Debido al \'exito conseguido con el sistema de reconocimiento de barras, el Comit\'e Ejecutivo de FIFA\circledR\ ha aceptado
otorgar a la Selecci\'on Argentina el beneficio de reforzar el equipo mediante el reemplazo de alguno de los
\emph{muertos} que se encargan de los apectos defensivos del equipo por un robot, siempre y cuando siga las Tres Leyes
de la Rob\'otica \cite{asimov}.\footnote{Se est\'a considerando \emph{darle hacha} a la rodilla de Neymar Jr. o de Alexis
S\'anchez en un eventual cruce con las selecciones hermanas.} Ante esta posibilidad, Sabella no dud\'o y se ha
contactado con el Equipo de Desarrollos de M\'etodos Num\'ericos para, esta vez, colaborar para traer \emph{la tan
ansiada copa} a la Argentina luego de
28 a\~nos, desde aquella alegr\'ia del Diego. El an\'alisis preliminar determin\'o que se deben estudiar y adaptar las t\'ecnicas utilizadas habitualmente
en el \emph{f\'utbol de robots} y desarrollar una herramienta que se llamar\'a \emph{Localized Trajectory Approximator}
(LTA).

Uno de los problemas m\'as importantes a resolver en este contexto es
predecir con la mayor anticipaci\'on posible la posici\'on futura de la 
pelota en funci\'on de su posici\'on en el pasado reciente. Sobre la base
de estas predicciones se coordinan los movimientos de los jugadores de
campo y, en el caso que nos ocupa ahora, la posici\'on del arquero cuando
existe peligro de gol.
Cuando la pelota se dirige hacia nuestro arco, es muy importante que
ubiquemos el arquero en la posici\'on exacta en la que la pelota 
cruzar\'a la l\'\i nea de gol, de manera que pueda interceptarla y evitar
la ca\'{\i}da de nuestra valla. El sistema de control de cada equipo
suministra informaci\'on en pasos discretos. En cada paso nuestras
c\'amaras de video determinan la posici\'on de la pelota, y debemos
indicarle la acci\'on a seguir al arquero.
%: quedarse quieto, moverse 
%hacia la izquierda o moverse hacia la derecha.

Los postes del arco est\'an ubicados en coordenadas fijas,
y la l\'\i nea de gol es el segmento entre estos dos puntos. Se marca
un gol cuando la pelota cruza este segmento. Vistos desde arriba, la pelota
es un c\'\i rculo de radio determinado y el arquero 
se representa mediante un segmento paralelo a la l\'\i nea de gol,
ubicado sobre la misma.
Inicialmente el arquero se encuentra en punto del plano, sobre la linea de gol, 
y en cada paso se le indica al arquero qu\'e acci\'on
debe tomar. Las posibles acciones en cada instante de tiempo son dos: moverse hacia alguno de los lados, izquierda o
derecha, una cantidad de pasos (acotada por un valor m\'aximo $\mu$), o quedarse quieto y no hacer nada.

Formalmente, consideraremos un horizonte discreto de tiempo $0, 1, \dots, T$ en el cual se realiza el disparo y describiremos
la trayectoria de la pelota mediante una funci\'on $p:\mathbb{R} \to \mathbb{R}^2$, $p(t) = (x(t),y(t))$ que permite
describir la posici\'on de la pelota en funci\'on del tiempo. Tanto el arco, como los jugadores
rivales y nuestro arquero se ubican en posiciones que representaremos con coordenadas $(x_i,y_i)$ en el plano
$\mathbb{R}^2$. En paticular, nuestro arquero s\'olo se mover\'a sobre el eje $y$, manteniendo fija su coordenada $x$ en
la l\'inea de gol. El objetivo del trabajo es en cada instante de timepo reportar un valor $z$, $-\mu \le z \le \mu$ que
indique cu\'anto y hacia d\'onde se desplaza el arquero respecto de la posici\'on actual.

Debido a la tecnolog\'ia utilizada para la detecci\'on de la pelota, un problema fundamental que debe enfrentarse es la presencia de ruido
en las mediciones de la posici\'on de la misma. El sistema de visi\'on
est\'a sujeto a vibraciones, golpes y errores de captura de datos,
que hacen que las mediciones de la pelota sufran errores.
Por otra parte, la pelota no siempre viaja hacia el arco en l\'\i nea
recta sino que puede describir curvas m\'as o menos complicadas,
dependiendo del ``efecto'' dado por el jugador al momento de impactar
la pelota y de posibles curvaturas en la superficie del campo de juego.

Adem\'as de la informaci\'on del arquero y la pelota, contaremos tambi\'en con las posiciones de los jugadores rivales
en la cancha. El objetivo de considerar esta informaci\'on adicional radica en simular una situaci\'on de juego real.
Para simplificar levemente el alcance en esta etapa asumiremos que la posici\'on de los jugadores no var\'ia con el
tiempo, aunque s\'i pueden intervenir en el juego y desv\'iar la pelota con un remate al arco o un pase a otro jugador.

{\bf\noindent Enunciado y competencia}

Se pide implementar un programa en \verb+C+ o \verb-C++- que resuelva el problema de determinar la posici\'on del
arquero utilizando t\'ecnicas y m\'etodos vistos en la materia. Para ello, dada la posici\'on incial del
arquero, los l\'imites del arco, las posiciones de los jugadores del equipo rival y la evoluci\'on de la posici\'on 
de la pelota (o su estimaci\'on) en distintos instantes de tiempo, deber\'a determinar qu\'e acci\'on debe realizar el
arquero en cada momento con el objetivo final de atajar el disparo. Si bien los datos pueden tener ruido debido al
sistema de captura, asumiremos que la \'ultima medici\'on carece de error y representa la posici\'on efectiva de la
pelota. 

El trabajo constar\'a de al menos un m\'etodo que pueda ser utilizado para predecir la posici\'on futura de la pelota,
implementado \'integramente por el grupo (incluyendo sus m\'etodos de resoluci\'on). En caso de
considerar m\'as de una alternativa, es posible desarrollar los restantes m\'etodos utilizando librer\'ias p\'ublicas (por
ejemplo, \emph{boost} \cite{BoostSite}). En todos los casos es imprescindible incluir el desarrollo una descripci\'on detallada
de los mismos, justificando adecuadamente su elecci\'on. Adem\'as de su explicaci\'on, es obligatorio que todos los m\'etodos propuestos 
sean debidamente evaluados y analizados experimentalmente, mediante la formulaci\'on de hip\'otesis, realizaci\'on de
experimentos y el posterior an\'alisis de los resultados obtenidos en los mismos, buscando caracterizar y justificar su
comportamiento en funci\'on del problema a resolver.

En el contexto del trabajo, los arqueros propuestos por los distintos grupos participar\'an, junto con un arquero
propuesto por la c\'atedra, en una competencia que se llevar\'a a cabo el d\'ia 27 de Junio de 2014, a las 18:00 hs,
luego de la devoluci\'on de los trabajos corregidos. El formato de la competencia, las instancias de prueba y los
premios ser\'an informados durante los primeros d\'ias de Junio. Es importante destacar que la aprobaci\'on del Trabajo
Pr\'actico es independiente de la competencia.

Para simplificar la implementaci\'on, el archivo de entrada contar\'a con toda
la informaci\'on del disparo, aunque para tomar la decisi\'on en el instante actual no se podr\'a utilizar informaci\'on
futura. Aquellos grupos que violen esta regla ser\'an autom\'aticamente reprobados y descalificados de la competencia.

\newpage

{\bf\noindent Programa y formato de archivos}

El programa a desarrollar deber\'a tomar por l\'inea de comandos un archivo de entrada con la descripci\'on del tiro y
generar un archivo de salida, cuyo nombre ser\'a tambi\'en indicado mediante un par\'ametro, con los movimientos a
realizar por el arquero en cada instante de tiempo.

El archivo de entrada seguir\'a el siguiente formato:
\begin{itemize}
\item La primera l\'inea contendr\'a la posici\'on inicial del arquero en $y$, y luego las coordenadas que definen los l\'imites del arco, tambi\'en sobre el eje $y$. Se asume que la posici\'on en $x$ del arquero y de la l\'inea de gol son las mismas: $x=125$. Finalmente estará $\mu$, la cota sobre el m\'aximo desplazamiento que puede realizar el arquero
en un instante de tiempo.
\item Luego se muestra la secuencia de posiciones en $\mathbb{R}^2$, una por l\'inea, que toma la pelota para los instantes de tiempo
$0,1,\dots,T$.
\end{itemize}

A continuaci\'on se muestra un archivo donde el arquero comienza en la posici\'on $(125,375)$, donde el arco tiene su
limite superior en la coordenada $(125,385)$ y la inferior en $(125,315)$. Recorda que el arquero siempre se encuentra
sobre la l\'inea de gol. Finalmente, $\mu = 4$, y luego se tiene una secuencia de $24$ posiciones con la definici\'on del
tiro para los instantes de tiempo $0,1,\dots,23$.

\begin{verbatim}
375 315 385 4
468.0 522.0
450.0 512.0
432.0 502.0
414.0 492.0
396.0 482.0
378.0 472.0
360.0 463.0
342.0 453.0
324.0 443.0
306.0 433.0
288.0 423.0
270.0 413.0
252.0 403.0
234.0 393.0
216.0 383.0
198.0 374.0
180.0 364.0
162.0 354.0
144.0 344.0
126.0 334.0
108.0 324.0
90.0 314.0
72.0 304.0
54.0 295.0
\end{verbatim}

El archivo de salida debera contener una instrucci\'on por l\'inea, correspondiente a la acci\'on que realiza el arquero
en el instante $0 \le t \le T$. Un posible resultado para el tiro descripto anteriormente es el siguiente, donde el
arquero no realiza ning\'un movimiento salvo en el ultimo instante donde se nueve a la posici\'on $(125,371)$.
\begin{verbatim}
0
0
0
0
0
0
0
0
0
0
0
0
0
0
0
0
0
0
0
0
0
0
0
-4
\end{verbatim}

Un detalle importante es cuando el visualizador considera si el disparo ha sido (o no) atajado por el arquero. Tanto para
el arquero como para la pelota, las posiciones inidicadas se corresponden con el centro de los mismos. Luego, si la
pelota se encuentra a una distancia menor o igual a 7 del centro del aquero, entonces consideramos que el disparo ha
sido atajado. En caso contrario, contar\'a como gol.

Junto con el presente enunciado, se adjunta una serie de scripts hechos en \verb+python+ con un visualizador y un conjunto instancias de
test para ser utilizados para la compilaci\'on y un testeo b\'asico de la implementaci\'on y los m\'etodos. En las
pr\'oximas semanas agregaremos m\'as archivos con distintos escenarios de prueba. El visualizador puede ser
utilizado para simular el disparo y evaluar si el programa propuesto logra que el arquero ataje la pelota. Este
visualizador ser\'a utilizado durante la competencia y es la herramienta que determina si un tiro ha
sido atajado o no. Se recomienda leer
el archivo \verb+README.txt+ con el detalle sobre su utilizaci\'on.

\vskip 0.5 cm
\hrule
\vskip 0.1 cm

{\bf Fecha de entrega} 
\begin{itemize}
\item \textsl{Formato electr\'onico:} Jueves 19 de Junio de 2014, \underline{hasta las 23:59 hs.}, enviando el trabajo
(informe+c\'odigo) a \texttt{metnum.lab@gmail.com}. El subject del email debe comenzar con el texto \verb|[TP3]| seguido
de la lista de apellidos de los integrantes del grupo. 
\item \textsl{Formato f\'isico:} Lunes 23 de Junio de 2014, de 17:30 a 18:00 hs.
\end{itemize}

\bibliographystyle{plain}
\bibliography{tp3}

\end{document}
